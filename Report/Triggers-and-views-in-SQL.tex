\documentclass{report}

% Paquetes y configuraciones adicionales
\usepackage{graphicx}
\usepackage[export]{adjustbox}
\usepackage{caption}
\usepackage{float}
\usepackage{titlesec}
\usepackage{geometry}
\usepackage{hyperref}


% Configura los márgenes
\geometry{
    left=2cm,   % Ajusta este valor al margen izquierdo deseado
    right=2cm,  % Ajusta este valor al margen derecho deseado
    top=3cm,
    bottom=3cm,
}

% Configuración de los títulos de las secciones
\titlespacing{\section}{0pt}{\parskip}{\parskip}
\titlespacing{\subsection}{0pt}{\parskip}{\parskip}
\titlespacing{\subsubsection}{0pt}{\parskip}{\parskip}


\begin{document}
	
	% Portada del informe
	
	\title{Disparadores y Vistas en SQL}
	\author{Samuel Martín Morales}
	\date{\today}
	
	\maketitle
	
	% Índice
	\tableofcontents
	
	% Secciones del informe
	\chapter{Introducción}
  Para esta cuarta práctica de la asignatura \emph{Administración y Diseño de Bases de Datos} se solicita el empleo de una base de datos que debe de ser restaurada de manera previa a la implementación de una serie de ejercicios que son demandados haciendo uso de dicha base de datos.\\

	En este caso, la base de datos a emplear se denomina como \emph{\textbf{alquilerdvd.tar}} y se encuentra disponible en el campus virtual de la asignatura. Pero, puede ser descargada desde el siguiente enlace de \href{https://github.com/Samuelmm15/PostgreSQL-Rent/blob/main/AlquilerPractica.tar}{GitHub}.\\

	Dicha base de datos se encuentra en formato \emph{.tar} por lo que, para poder restaurarla, se debe de emplear el siguiente comando:

	\begin{verbatim}
		$ pg_restore -U postgres -d alquilerdvd alquilerdvd.tar
	\end{verbatim}

	Es decir, el comando anterior restaura la base de datos \emph{alquilerdvd} haciendo uso del fichero \emph{alquilerdvd.tar} y empleando el usuario \emph{postgres}.\\

	Una vez restaurada la base de datos, se puede proceder a la realización de los distintos ejercicios.\\

	\chapter{Resultados}
	\section{Ejercicio 1}
	
	
	\chapter{Conclusiones}
  Example....	

	\chapter{Bibliografía}
	Example....
	
\end{document}